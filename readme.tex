\documentclass[a4paper,12pt]{article}
\usepackage[margin=1in]{geometry}
\usepackage{enumerate}

\begin{document}

\section{Basic use}
To run program on a super computer requires the following modules, or something equivalent
\begin{itemize}
	\item gcc/gcc-4.9.2
	\item openmpi/openmpi-1.8.4\_gcc-4.9.2
	\item slurm/slurm
	\item gsl/gsl-1.16\_gcc-4.9.2
\end{itemize}
Use \textit{make -f parallel} to make the parallel executables, \textit{./EurPar} and \textit{./EncPar}.
To run code, submit a batch script with desired parameters.


Output files for Jet \# are saved in \\\\
\indent /lustre/janus\_scratch/beso3770/Eur\_JetResults/Jet\#/ \\
\indent /lustre/janus\_scratch/beso3770/Enc\_JetResults/Jet\#/ \\\\
Note that these folders are specific to Janus, the CU supercomputer. On other computers
the output address must be changed in Jet.cpp. Once binary files have been produced, 
zip up binary files for download via, e.g., \textit{zip Jet1\_dens.zip ./*.dens}.


General notes -
	- C-code reads in normal longitude (LON). Note that *western longitude* (WLON)
		+ LON = 360 - mod(WLON,360)
	- Many things can be specified via flags on command line
	- Some things like what sizes to simulate must be changed in the code
	- Large particles - simulated grid with 50m resolution, height 15km and width 25km for particles sized 


\section{Structure}


\end{document}
